% !TeX root = main.tex
% !TeX program = pdflatex
\documentclass[11pt]{article}

\usepackage[T1]{fontenc}
\usepackage[utf8]{inputenc}
\usepackage[a4paper,margin=1in]{geometry}
\usepackage{amsmath,amssymb,amsthm,mathtools}
\usepackage{enumitem}
\usepackage{hyperref}
\hypersetup{colorlinks=true, linkcolor=black, urlcolor=blue}

\newtheorem{theorem}{Theorem}
\newtheorem{lemma}[theorem]{Lemma}
\theoremstyle{definition}
\newtheorem{definition}[theorem]{Definition}
\theoremstyle{remark}
\newtheorem*{remark}{Remark}

\newcommand{\N}{\mathbb{N}}
\newcommand{\R}{\mathbb{R}}
\newcommand{\dist}{\mathrm{dist}}

\begin{document}

\begin{center}
{\Large \textbf{Written Assignment 1 — Solutions}}\\[0.25em]
CS 440 \quad \today
\end{center}

\noindent\textbf{Name: Yat Long Szeto} \quad
\textbf{BU ID:  90479281} 

\bigskip

\section*{Question 1: Shortest Path Composition}

\noindent\textbf{Answer}\\
\begin{align*}
p^\ast &= p_1 \cdot p_2 &\text{split the $a\!\to\! b$ path at $c$ (concatenation)}\\
\text{cost}(p^\ast) &= \text{cost}(p_1) + \text{cost}(p_2) &\text{path cost additivity}\\
\exists\,\tilde p_1:\ \text{cost}(\tilde p_1) &< \text{cost}(p_1) &\text{assumption for contradiction}\\
\text{cost}(\tilde p_1\!\cdot\! p_2) &= \text{cost}(\tilde p_1) + \text{cost}(p_2) &\text{additivity}\\
&< \text{cost}(p_1) + \text{cost}(p_2) &\text{by assumption}\\
&= \text{cost}(p^\ast) &\text{from the first two lines}
\end{align*}

Since $\text{cost}(\tilde p_1\!\cdot\! p_2) < \text{cost}(p^\ast)$, 
this contradicts that $p^\ast$ is a shortest $a\!\to\! b$ path. 
Hence $p_1$ must be shortest $a\!\to\! c$. By the same argument, $p_2$ is shortest $c\!\to\! b$.


\section*{Question 2: Iterative Deepening Returns Shortest Paths}

\begin{theorem}
On an unweighted (unit-edge) graph, Iterative Deepening DFS with limits $1,2,3,\dots$ from a source $s$ returns, for any discovered vertex $v$, a path with the minimum number of edges.
\end{theorem}

\noindent\textbf{Answer}\\
\begin{align*}
\ell(v) &= \min\{\text{length}(P): P \text{ is an $s\!\to\! v$ path}\} &\text{define $\ell(v)$ as shortest distance}\\
d &< \ell(v) \ \Rightarrow \ \text{no $s\!\to\! v$ path found} &\text{depth bound too small}\\
d &= \ell(v) \ \Rightarrow \ \text{a shortest $s\!\to\! v$ path is eligible} &\text{length $\ell(v)$ now allowed}\\
\text{algorithm completes all of depth }d &= \ell(v) \text{ before } d=\ell(v)+1 &\text{order of iterative deepening}\\
\Rightarrow\ v \text{ first discovered at depth } &= \ell(v) &\text{cannot appear earlier, appears now}\\
\text{thus path returned has length } &= \ell(v) &\text{definition of shortest path length}
\end{align*}

\section*{Question 3: Diameter bound}
\begin{theorem}
For any finite, undirected, unweighted graph $G=(V,E)$, the diameter satisfies $\mathrm{diam}(G)\le |V|-1$.
\end{theorem}

\begin{proof}
Fix $u\neq v$. Any \emph{simple} $u$–$v$ path visits vertices at most once, hence uses at most $|V|-1$ edges. A shortest $u$–$v$ path is simple (otherwise remove a cycle to shorten it). So $\dist(u,v)\le |V|-1$ for all pairs, and the maximum over pairs is at most $|V|-1$.
\end{proof}

\section*{Question 4: Correctness of Dijkstra's algorithm\\(positive weights)}
\begin{theorem}
Let $G=(V,E,w)$ be directed with $w(e)>0$. Dijkstra’s algorithm from source $s$ settles each vertex $v$ with key $d[v]=\delta(s,v)$, the true shortest-path distance.
\end{theorem}

\begin{proof}[Loop invariant proof]
\emph{Invariant.} (i) For every settled $u$, $d[u]=\delta(s,u)$. (ii) For every fringe $x$, $d[x]$ equals the minimum length of an $s$–$x$ path whose last edge enters from a settled predecessor.

Initially only $s$ may be settled with $d[s]=0=\delta(s,s)$; relaxations preserve (ii). Suppose before an iteration the invariant holds and the algorithm extracts $v$ with minimal $d[v]$ among unsettled vertices. If $d[v]>\delta(s,v)$, take a shortest $s$–$v$ path and let $y$ be the first unsettled vertex on it with settled predecessor $x$. Then $d[x]=\delta(s,x)$ by (i), and relaxing $(x,y)$ gave $d[y]\le \delta(s,y)\le \delta(s,v)$. Minimality of $d[v]$ implies $d[v]\le d[y]\le \delta(s,v)$, a contradiction. Hence $d[v]=\delta(s,v)$ when settled; relaxations maintain (ii). By induction, the claim holds for all settled vertices.
\end{proof}

\bigskip
\section*{Extra Credit (outline): Column-constrained top-to-bottom shortest path on a vertex-weighted lattice}

\paragraph{Model.} $n\times n$ directed lattice; edges have weight $0$; each vertex $u$ has weight $w(u)>0$; path length is the sum of vertex weights (including the start).

\paragraph{Goal.} For each column $i$, let $P_i$ be a shortest path from the top vertex $v_{1,i}$ to bottom vertex $v_{n,i}$. Output $P^\ast=\arg\min_i \mathrm{len}(P_i)$.

\paragraph{Reduction using the given oracle.}
Augment $G$ with super-source $s$ and super-sink $t$ of zero weight. Connect $s\to v_{1,i}$ for all $i$. To enforce that the path ends in the \emph{same} column, build, for each fixed $i$, a graph $G^{(i)}$ that includes only the edge $v_{n,i}\to t$ (delete $v_{n,j}\to t$ for $j\neq i$). One call to the oracle on $(G^{(i)},s,t)$ returns $P_i$. Take the best over $i$. This takes $O(n)$ oracle calls on $\Theta(n^2)$-size lattices (overall $O(n^3)$). If lateral moves cannot change the column at the bottom, a single call on the graph with all $v_{n,i}\to t$ suffices in $O(n^2)$.

\paragraph{Correctness.} In $G^{(i)}$, any $s$–$t$ path must start at some $v_{1,i}$ and end at $v_{n,i}$; its cost equals the vertex-sum along the $v_{1,i}\leadsto v_{n,i}$ segment, i.e., the length of $P_i$. Minimizing over $i$ yields $P^\ast$.

\end{document}
