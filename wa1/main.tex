% !TeX root = main.tex
% !TeX program = pdflatex
\documentclass[11pt]{article}

\usepackage[T1]{fontenc}
\usepackage[utf8]{inputenc}
\usepackage[a4paper,margin=1in]{geometry}
\usepackage{amsmath,amssymb,amsthm,mathtools}
\usepackage{enumitem}
\usepackage{hyperref}
\hypersetup{colorlinks=true, linkcolor=black, urlcolor=blue}

\newtheorem{theorem}{Theorem}
\newtheorem{lemma}[theorem]{Lemma}
\theoremstyle{definition}
\newtheorem{definition}[theorem]{Definition}
\theoremstyle{remark}
\newtheorem*{remark}{Remark}

\newcommand{\N}{\mathbb{N}}
\newcommand{\R}{\mathbb{R}}
\newcommand{\dist}{\mathrm{dist}}

\begin{document}

\begin{center}
{\Large \textbf{Written Assignment 1 — Solutions}}\\[0.25em]
CS 440 \quad \today
\end{center}

\noindent\textbf{Name: Yat Long Szeto} \quad
\textbf{BU ID:  90479281} 

\bigskip

\section*{Question 1: Shortest Path Composition}

\noindent\textbf{Proof}\\
\begin{align*}
p^\ast &= p_1 \cdot p_2 &\text{split $a\!\to\! b$ path at $c$}\\
\text{cost}(p^\ast) &= \text{cost}(p_1) + \text{cost}(p_2) &\text{path cost additivity}\\
\exists\,\tilde p_1:\ \text{cost}(\tilde p_1) &< \text{cost}(p_1) &\text{assume $p_1$ not shortest}\\
\text{cost}(\tilde p_1 \cdot p_2) &= \text{cost}(\tilde p_1) + \text{cost}(p_2) &\text{concatenate paths}\\
&< \text{cost}(p_1) + \text{cost}(p_2) &\text{by assumption}\\
&= \text{cost}(p^\ast) &\text{substitution}
\end{align*}

This contradicts the optimality of $p^\ast$.
Therefore $p_1$ must be a shortest $a\!\to\! c$ path and there is no such $\tilde p_1$ exists.  
WLOG, replace the prefix $p_1$ by the suffix $p_2$ and $c$ by $b$. 
The same reasoning shows that if $p_2$ were not shortest $c\!\to\! b$, 
then replacing it with a shorter path would also contradict the optimality of $p^\ast$. 
Hence $p_2$ is a shortest $c\!\to\! b$ path as well.

\section*{Question 2: Iterative Deepening Returns Shortest Paths}

\noindent\textbf{Proof}\\
\begin{align*}
\ell(v) &= \min\{\text{length}(P): P \text{ is an $s\!\to\! v$ path}\} &\text{define shortest distance}\\
\text{When IDDFS, } d < \ell(v) &\;\Rightarrow\; \text{$v$ not found} &\text{too shallow}\\
d = \ell(v) &\;\Rightarrow\; \text{$v$ reachable} &\text{path length fits}\\
\text{IDDFS finishes depth } d &= \ell(v) \text{ before } d+1 &\text{order of search}\\
v \text{ first discovered} &= \text{ at depth } \ell(v) &\text{cannot appear earlier}\\
\end{align*}

Therefore, the first time $v$ is returned by Iterative Deepening is exactly at depth $\ell(v)$, and the path has the minimum number of edges.


\section*{Question 3: Diameter Bound}

\noindent\textbf{Proof}\\
\begin{align*}
\text{Fix }u,v\in V,\ u&\neq v. &\text{setup}\\
W &= (v_0,\ldots,v_k),\ v_0=u,\ v_k=v &\text{a $u\!\to\! v$ walk of minimum length}\\
\exists\, i<j:\ v_i&=v_j \ \Rightarrow\ W'=(v_0,\ldots,v_i,v_{j+1},\ldots,v_k) &\text{delete cycle}\\
\operatorname{len}(W')&=k-(j-i) < k=\operatorname{len}(W) &\text{strictly shorter}\\
\Rightarrow\ \neg\exists\, i<j:\ v_i&=v_j &\text{contradiction to minimality}\\
\Rightarrow\ W&\text{ is simple} &\text{no repetitions}\\[3pt]
\text{Let }P=W,\ |P|&=m\ \text{(edges)} \ \Rightarrow\ \text{$P$ visits }m+1\text{ distinct vertices} &\text{path has one more vertex}\\
m+1 &\le |V| \ \Rightarrow\ m\le |V|-1 &\text{counting bound}\\
\operatorname{dist}(u,v) &= |P| \le |V|-1 &\text{shortest path equals $|P|$}\\
\operatorname{diam}(G) &= \max_{u\neq v}\operatorname{dist}(u,v) \le |V|-1 &\text{take max}
\end{align*}



\section*{Question 4 — Dijkstra’s Algorithm and Shortest Paths}

\begin{theorem}
Let $G=(V,E,w)$ be a directed graph with strictly positive edge weights $w(e)>0$. When Dijkstra’s algorithm is run from a source $s$ and it returns a path to any vertex $v$, the returned path has length $\delta(s,v)$, the true shortest-path distance from $s$ to $v$.
\end{theorem}

\begin{proof}
Assume for contradiction that some vertex is \emph{settled} with an incorrect label. 
Let $v$ be the first such vertex extracted from the priority queue with $d[v]>\delta(s,v)$;
 thus, every previously settled $u$ satisfies $d[u]=\delta(s,u)$.

Consider a shortest $s\!\to\! v$ path $P$, and let $y$ be the first vertex on $P$ that is not yet settled just before $v$ is extracted; let $x$ be the predecessor of $y$ on $P$. By choice of $y$, $x$ is settled. By the induction hypothesis for earlier settled vertices, $d[x]=\delta(s,x)$. When $x$ was settled, the relaxation of edge $(x,y)$ gave
\[
d[y]\ \le\ d[x]+w(x,y)\ =\ \delta(s,x)+w(x,y)\ =\ \delta(s,y),
\]
and since $y$ lies on a shortest $s\!\to\! v$ path, we have $\delta(s,y)\le \delta(s,v)$. Hence
\[
d[y]\ \le\ \delta(s,v).
\]
Because Dijkstra extracts the unsettled vertex with minimum key, it holds that
\[
d[v]\ \le\ d[y]\ \le\ \delta(s,v),
\]
contradicting the assumption $d[v]>\delta(s,v)$. Therefore no such $v$ exists, and every vertex is settled with its true distance; in particular, the path returned to any $v$ is a shortest path.
\end{proof}


\end{document}
