% !TeX root = main.tex
% !TeX program = pdflatex
\documentclass[11pt]{article}

\usepackage[T1]{fontenc}
\usepackage[utf8]{inputenc}
\usepackage[a4paper,margin=1in]{geometry}
\usepackage{amsmath,amssymb,amsthm,mathtools}
\usepackage{enumitem}
\usepackage{hyperref}
\hypersetup{colorlinks=true, linkcolor=black, urlcolor=blue}

\newtheorem{theorem}{Theorem}
\newtheorem{lemma}[theorem]{Lemma}
\theoremstyle{definition}
\newtheorem{definition}[theorem]{Definition}
\theoremstyle{remark}
\newtheorem*{remark}{Remark}

\newcommand{\N}{\mathbb{N}}
\newcommand{\R}{\mathbb{R}}
\newcommand{\dist}{\mathrm{dist}}

\begin{document}

\begin{center}
{\Large \textbf{Written Assignment 1 — Solutions}}\\[0.25em]
CS 440 \quad \today
\end{center}

\noindent\textbf{Name: Yat Long Szeto} \quad
\textbf{BU ID:  90479281} 

\bigskip

\section*{Question 1: Shortest Path Composition}

\begin{proof}
When considering the path from $a$ to $c$,
suppose $p_1$ is not shortest from $a$ to $c$. 
Then there exists an $a$ to $c$ path $\tilde p_1$ with $cost(\tilde p_1) < cost(p_1)$. 
Concatenate to get an $a$ to $b$ walk $\tilde p_1\cdot p_2$ of cost
\[
cost(\tilde p_1)+cost(p_2) < cost(p_1)+cost(p_2)=cost(p^\ast),
\]
contradicting optimality of $p^\ast$ as the cost of $\tilde p_1\cdot p_2$ is less than $p^\ast$.
 Hence $p_1$ is shortest $a\!\to\! c$ and there is no such $\tilde p_1$ exists.
 The same argument with $(c,b)$ shows $p_2$ is shortest $c\!\to\! b$.
\end{proof}

\section*{Question 2: Iterative Deepening returns shortest paths}

\begin{proof}
Let $\ell(v)$ be the shortest-path distance from $s$ to $v$, for some source node $s$ and some vertex $v$.
For some depth limit d such $d<\ell(v)$, depth-limited DFS cannot expose any $s$–$v$ path.
When $d=\ell(v)$, a shortest $s$–$v$ path of length $\ell(v)$ is eligible and will be found during that pass ($d=\ell(v)$). 
Therefore the first time $v$ is returned is at depth limit $\ell(v)$, with a shortest path.

\end{proof}

\section*{Question 3: Diameter bound}

\begin{proof}
For any $u\neq v$, any simple $u$–$v$ path visits vertices at most once, 
hence uses at most $|V|-1$ edges.
 A shortest $u$–$v$ path is simple (otherwise remove a cycle to shorten it). 
 So $\dist(u,v)\le |V|-1$ for all pairs, and the maximum over pairs is at most $|V|-1$.
 The diameter is at most $|V|-1$.
\end{proof}

\section*{Question 4 — Dijkstra’s Algorithm and Shortest Paths}

\begin{theorem}
Let $G=(V,E,w)$ be a directed graph with strictly positive edge weights $w(e)>0$. When Dijkstra’s algorithm is run from a source $s$ and it returns a path to any vertex $v$, the returned path has length $\delta(s,v)$, the true shortest-path distance from $s$ to $v$.
\end{theorem}

\begin{proof}
Assume for contradiction that some vertex is \emph{settled} with an incorrect label. 
Let $v$ be the first such vertex extracted from the priority queue with $d[v]>\delta(s,v)$;
 thus, every previously settled $u$ satisfies $d[u]=\delta(s,u)$.

Consider a shortest $s\!\to\! v$ path $P$, and let $y$ be the first vertex on $P$ that is not yet settled just before $v$ is extracted; let $x$ be the predecessor of $y$ on $P$. By choice of $y$, $x$ is settled. By the induction hypothesis for earlier settled vertices, $d[x]=\delta(s,x)$. When $x$ was settled, the relaxation of edge $(x,y)$ gave
\[
d[y]\ \le\ d[x]+w(x,y)\ =\ \delta(s,x)+w(x,y)\ =\ \delta(s,y),
\]
and since $y$ lies on a shortest $s\!\to\! v$ path, we have $\delta(s,y)\le \delta(s,v)$. Hence
\[
d[y]\ \le\ \delta(s,v).
\]
Because Dijkstra extracts the unsettled vertex with minimum key, it holds that
\[
d[v]\ \le\ d[y]\ \le\ \delta(s,v),
\]
contradicting the assumption $d[v]>\delta(s,v)$. Therefore no such $v$ exists, and every vertex is settled with its true distance; in particular, the path returned to any $v$ is a shortest path.
\end{proof}


\end{document}
