% !TeX root = main.tex
% !TeX program = pdflatex
\documentclass[11pt]{article}

\usepackage[T1]{fontenc}
\usepackage[utf8]{inputenc}
\usepackage[a4paper,margin=1in]{geometry}
\usepackage{amsmath,amssymb,amsthm,mathtools}
\usepackage{enumitem}
\usepackage{hyperref}
\hypersetup{colorlinks=true, linkcolor=black, urlcolor=blue}

\newtheorem{theorem}{Theorem}
\newtheorem{lemma}[theorem]{Lemma}
\theoremstyle{definition}
\newtheorem{definition}[theorem]{Definition}
\theoremstyle{remark}
\newtheorem*{remark}{Remark}

\newcommand{\N}{\mathbb{N}}
\newcommand{\R}{\mathbb{R}}
\newcommand{\dist}{\mathrm{dist}}

\begin{document}

\begin{center}
{\Large \textbf{Written Assignment 1 — Solutions}}\\[0.25em]
CS 440 \quad \today
\end{center}

\noindent\textbf{Name: Yat Long Szeto} \quad
\textbf{BU ID:  90479281} 

\bigskip

\section*{Question 1: Shortest Path Composition}

\noindent\textbf{Proof}\\
\begin{align*}
p^\ast &= p_1 \cdot p_2 &\text{split $a\!\to\! b$ path at $c$}\\
\text{cost}(p^\ast) &= \text{cost}(p_1) + \text{cost}(p_2) &\text{path cost additivity}\\
\exists\,\tilde p_1:\ \text{cost}(\tilde p_1) &< \text{cost}(p_1) &\text{assume $p_1$ not shortest}\\
\text{cost}(\tilde p_1 \cdot p_2) &= \text{cost}(\tilde p_1) + \text{cost}(p_2) &\text{concatenate paths}\\
&< \text{cost}(p_1) + \text{cost}(p_2) &\text{by assumption}\\
&= \text{cost}(p^\ast) &\text{substitution}
\end{align*}

This contradicts the optimality of $p^\ast$.
Therefore $p_1$ must be a shortest $a\!\to\! c$ path and there is no such $\tilde p_1$ exists.  
WLOG, replace the prefix $p_1$ by the suffix $p_2$ and $c$ by $b$. 
The same reasoning shows that if $p_2$ were not shortest $c\!\to\! b$, 
then replacing it with a shorter path would also contradict the optimality of $p^\ast$. 
Hence $p_2$ is a shortest $c\!\to\! b$ path as well.

\section*{Question 2: Iterative Deepening Returns Shortest Paths}

\noindent\textbf{Proof}\\
\begin{align*}
\ell(v) &= \min\{\text{length}(P): P \text{ is an $s\!\to\! v$ path}\} &\text{define shortest distance}\\
\text{When IDDFS, } d < \ell(v) &\;\Rightarrow\; \text{$v$ not found} &\text{too shallow}\\
d = \ell(v) &\;\Rightarrow\; \text{$v$ reachable} &\text{path length fits}\\
\text{IDDFS finishes depth } d &= \ell(v) \text{ before } d+1 &\text{order of search}\\
v \text{ first discovered} &= \text{ at depth } \ell(v) &\text{cannot appear earlier}\\
\end{align*}

Therefore, the first time $v$ is returned by Iterative Deepening is exactly at depth $\ell(v)$, and the path has the minimum number of edges.


\section*{Question 3: Diameter Bound}

\noindent\textbf{Proof}\\
\begin{align*}
\text{Fix }u,v\in V,\ u&\neq v. &\text{setup}\\
W &= (v_0,\ldots,v_k),\ v_0=u,\ v_k=v &\text{a $u\!\to\! v$ walk of minimum length}\\
\exists\, i<j:\ v_i&=v_j \ \Rightarrow\ W'=(v_0,\ldots,v_i,v_{j+1},\ldots,v_k) &\text{delete cycle}\\
\operatorname{len}(W')&=k-(j-i) < k=\operatorname{len}(W) &\text{strictly shorter}\\
\Rightarrow\ \neg\exists\, i<j:\ v_i&=v_j &\text{contradiction to minimality}\\
\Rightarrow\ W&\text{ is simple} &\text{no repetitions}\\[3pt]
\text{Let }P=W,\ |P|&=m\ \text{(edges)} \ \Rightarrow\ \text{$P$ visits }m+1\text{ distinct vertices} &\text{path has one more vertex}\\
m+1 &\le |V| \ \Rightarrow\ m\le |V|-1 &\text{counting bound}\\
\operatorname{dist}(u,v) &= |P| \le |V|-1 &\text{shortest path equals $|P|$}\\
\operatorname{diam}(G) &= \max_{u\neq v}\operatorname{dist}(u,v) \le |V|-1 &\text{take max}
\end{align*}



\section*{Question 4: Dijkstra’s Algorithm and Shortest Paths}

\noindent\textbf{Notation.}  
For vertices $u,v \in V$:  
\[
\delta(u,v) = \min \{\text{cost}(P) : P \text{ is a path from }u\text{ to }v\}
\]
denotes the true shortest-path cost.  
Dijkstra maintains tentative labels $d[v]$ which are upper bounds on $\delta(s,v)$.  
At each step, the algorithm extracts the vertex $v$ with minimal $d[v]$ among unvisited vertices.

\noindent\textbf{Answer}\\
\begin{align*}
\forall v:\ d[v] &\ge \delta(s,v) &\text{tentative labels are upper bounds}\\
\text{Suppose } d[v] &> \delta(s,v) &\text{assume contradiction}\\
\exists P &: s=x_0,\ldots,x_k=v,\ \text{cost}(P)=\delta(s,v) &\text{true shortest path}\\
\text{Let }x_j &= \text{first vertex of $P$ not yet settled} &\text{prefix in settled set $S$}\\
x_{j-1}\in S &: d[x_{j-1}]=\delta(s,x_{j-1}) &\text{inductive assumption}\\
d[x_j] &\le d[x_{j-1}]+w(x_{j-1},x_j) &\text{relaxation}\\
&= \delta(s,x_{j-1})+w(x_{j-1},x_j)\\
&= \delta(s,x_j) &\text{shortest-path property}\\
\Rightarrow\ d[x_j] &\le \delta(s,x_j) < \delta(s,v) < d[v] &\text{contradiction}
\end{align*}

Thus $d[v]=\delta(s,v)$ when $v$ is extracted. By induction over the extraction steps, every vertex is settled with its true shortest-path distance.

\medskip
\noindent\textbf{Conclusion.}  
When Dijkstra returns a path from $s$ to $v$, it is guaranteed to be the shortest path because:  
1. vertices are processed in nondecreasing order of distance,  
2. once extracted, $d[v]=\delta(s,v)$,  
3. no unvisited vertex can have a smaller true distance.  



\end{document}
